% !TeX root = RJwrapper.tex
\title{mpitbR: A toolbox for calculating multidimensional poverty indices in R}


\author{by Ignacio Girela}

\maketitle

\abstract{%
Multidimensional poverty measurement is a vital tool for assessing the deprivations and well-being of people across different dimensions, such as health, education, and living standards. However, existing packages for multidimensional poverty measurement and estimation have some limitations, such as ignoring the complex survey design of microdata or user-friendliness. In this paper, we present \CRANpkg{mpitbR}, a package for calculating multidimensional poverty indices based on the Alkire-Foster measurement approach, which accounts for the survey design, and offer various options and features for users. This package is the R version of the Stata \texttt{mpitb} package, which reproduces the workflow of the global Multidimensional Poverty Index developed by the United Nations Development Programme and Oxford University. The package provides functions for estimating the Alkire-Foster measures, as well as for analyzing poverty changes over time. The usage of the main functions and features of the \pkg{mpitbR} package are described and illustrated with an application over a synthetic data that has a typical household survey design.
}

\hypertarget{introduction}{%
\section{Introduction}\label{introduction}}

The first goal of the 2030 Agenda for Sustainable Development
ambitiously postulates \emph{Ending poverty in all its forms everywhere}.
This call to global action recognizes poverty as not merely a lack of
income but as a complex phenomenon with multiple dimensions that affect
individuals' well-being.

To capture these diverse aspects of poverty, the last two decades have
witnessed the development of various methodologies for multidimensional
poverty measurement. Notably, the \emph{dual-cut-off-counting} approach,
proposed by Alkire and Foster (2011) (AF), has gained widely adoption in both academic
research and policy-making spheres. The global Multidimensional Poverty
Index (MPI), published annually by the United Nations Development
Programme (UNDP) and the Oxford Poverty and Human Development Initiative
(OPHI), exemplifies the relevance of the AF method (UNDP and OPHI 2022).
Furthermore, numerous countries have built their national MPIs based on
the AF framework to monitor and evaluate their poverty reduction
strategies.

The AF method is lauded for its simplicity, comprehensiveness and
flexibility for measuring multidimensional poverty. An index derived
from the AF method can be disaggregated into several partial indices,
known as AF measures, which are invaluable for policy analysis and
design (Alkire 2021). Notwithstanding, calculating these measures in
practice is far from straightforward. Poverty measurement typically
depends on household survey data, and overlooking the complex survey
design can result in biased estimates, leading to erroneous statistical
inference, such as when comparing poverty levels across subgroups or
changes over time.

Therefore, there is a need for a user-friendly and reliable tool that
can estimate the AF measures and their associated standard errors,
confidence intervals, and p-values, taking into account the survey
design. To address this need, OPHI developed a Stata package called
\emph{mpitb}, which provides an integrated framework that mirrors the
estimation process of the global MPI to researchers, analysts, and
practitioners (Suppa 2023). This package offers a set of subcommands
for estimating the key quantities of interest and storing them in a
standardized format for further analysis. The main objective of this
paper is to present a package that adapts the mpitb framework for R
users. This package, named after \CRANpkg{mpitbR}, enables users to
estimate the AF measures and their standard errors and confidence
intervals, using methods from the \CRANpkg{survey} package
(Lumley 2004). In addition, with the view of consistency, the
\pkg{mpitbR} package provides functions with outputs that are equivalent
to the Stata package subcommands.

The main contribution of this paper is to introduce the \pkg{mpitbR}
package as a novel and useful tool for multidimensional poverty
measurement and analysis in R. Although there exists some R packages
available that compute AF measures, such as \CRANpkg{MPI} and
\CRANpkg{mpindex}, these packages do not account for the complex survey
design. They assume the data was obtained through a simple random
sampling, which is rarely the case of household surveys. The paper
focuses on demonstrating the functionality and features of the package
through a series of examples and applications using a household survey
data structure. It will be shown how the \pkg{mpitbR} package is
consistent with the global MPI workflow and that it can handle different
types of data and survey designs.

The remainder of this paper is organized as follows. Section 2 reviews
the AF method and the MPI calculation. Section 3 describes the
\pkg{mpitbR} package and its main functions. Section 4 illustrates the
usage of the package with some examples and applications. Section 6
concludes with a summary and provides some suggestions for further
extensions of the package.

\hypertarget{measuring-multidimensional-poverty-the-alkire-foster-method}{%
\section{Measuring multidimensional poverty: the Alkire-Foster Method}\label{measuring-multidimensional-poverty-the-alkire-foster-method}}

Alkire and Foster (2011) proposed a flexible approach to measure multidimensional poverty
that can be tailored to different contexts and policy purposes. The
flexibility of this method mainly derived from the so-called ``dual
cutoff counting approach'' for identifying the poor and the possibility
to build the MPI by aggregating different partial measures. Building an
MPI based on the AF approach can be summarized in the following steps
(for a detailed description, see Alkire et al. 2015):

\begin{enumerate}
\def\labelenumi{\arabic{enumi}.}
\item
  Determine a set of dimensions of poverty \(\mathcal{D}\) that are
  considered relevant for human development in a specific context
  (e.g., the global MPI chooses dimensions of health, education, and
  living standards, but other dimensions can be chosen depending on
  the context and goals).
\item
  Select \(d\) indicators that represents deprivations in each dimension
  (e.g., child mortality and malnutrition are the two indicators that
  represents health dimension in the global MPI).\\
\item
  Assign weights to each dimension and indicator, reflecting their
  relative importance where \(w_j\) is represents the weight of the
  \(j\)-th indicator for \(j = 1,\ldots,d\). In practice, indicators in
  each dimension are weighted equally such that
  \(\sum_{j=1}^d w_j = 1\).
\item
  Set the indicators deprivation cutoffs, which define the minimum
  level of achievement required to be considered non-deprived in each
  indicator (e.g., the global MPI uses cutoffs of having at least five
  years of schooling, having access to electricity, etc., but
  different thresholds can be set reflecting desired standards).
\item
  Apply the deprivations cutoff vector to each of the \(n\) observations
  (individuals or households) and build the \(n \times d\) deprivation
  matrix \(\mathbf{g^0}\). Each element
  \(\left[ \mathbf{g^0} \right]_{ij}\) of this matrix is a binary
  variable. If \(\left[ \mathbf{g^0} \right]_{ij} = 1\), the \(i\)-th
  observation is deprived in indicator \(j\), and the opposite if
  \(\left[ \mathbf{g^0} \right]_{ij} = 0\).
\item
  Build the weighted deprivation matrix \(\mathbf{\bar{g}^0}\) assigning
  the corresponding weight to each indicator (i.e.,
  \(\mathbf{\bar{g}^0} = \mathbf{g^0} \times \textbf{diag}(w_1,\ldots,w_d)\))
  and calculate the deprivations score for each observation \(c_i\),
  which is the weighted sum of the deprivations
  \(c_{i} = \sum_{j=1}^d w_j d_{ij}\).
\item
  Identify who is poor by setting a unique poverty cutoff \(k\) meaning
  the minimum proportion of weighted deprivations a household needs to
  experience to be considered multidimensional poor. This cutoff is
  compared with the deprivations score. Therefore, if \(c_{i} \geq k\),
  the person is multidimensional poor (e.g., the global MPI uses a
  cutoff \(k\) of 33.3\% or 1/3 which means that a person is poor if it
  is deprived in one-third or more of the weighted indicators).
\item
  Censor data of the non-poor and get the so-called censored
  (weighted) deprivation matrix (\(\mathbf{\bar{g}^0}(k)\)), and
  censored deprivations scores (\(c(k)\)), where \(c_{i} (k) = c_{i}\) if
  \(c_{i} \geq k\), and \(c_{i} (k) = 0\) otherwise.
\item
  Compute the MPI (\(M_0\)) by taking the mean of the censored
  deprivation score (\(c(k)\)):

  \[M_0 = \frac{1}{n} \sum_{i=1}^n c_i(k)\]

  As mentioned above, the MPI can be re-expressed as function of other
  partial measures which we referred as AF measures. For instance,
\end{enumerate}

\begin{itemize}
\item
  From the latter expression, by multiplying and dividing by the
  number of people identified as poor (\(q\))
  \[M_0 = \frac{q}{n} \times \frac{1}{q}\sum_{i=1}^n c_i(k) = H \times A \]
  we obtained the intensity (\(H\)) and intensity (\(A\)) of poverty. The
  former is the proportion of multidimensional poor people while the
  latter are the average weighted deprivations suffered by the poor.
\item
  From the censored deprivation matrix \(\mathbf{g^0}(k)\), if we take
  the mean of each column, we obtain the censored indicators headcount
  ratios \(h_j(k)\) which mean the proportion of people deprived in
  indicator \(j\) \emph{and} are multidimensional poor for \(j,\ldots,d\).
  From these measures we can also arrive to the MPI:
  \[M_0 = \sum_{j=1}^{d} w_j h_j(k)  \] Naturally, we can obtain the
  uncensored indicators headcount ratios \(h_j\) before censoring the
  non-poor from data \(\mathbf{g^0}\).
\item
  Also we can decompose not only the MPI but also every partial
  measure (\(H\), \(A\), \(h_j\), \(h_j(k)\)) by different population
  subgroups (age, regions, etc.).
\end{itemize}

\[ M_0 = \sum_{l=1}^L \phi^l M_0^l\] where \(\phi^l\) is the population
share of the \(l\)-th subgroup for \(l =1, \ldots, L\).

Finally, the absolute \(w_j \, h_j(k)\) and proportional
\(w_j \, h_j(k)/M_0\) contribution of each indicator to poverty are
usually reported.

\hypertarget{changes-over-time}{%
\subsubsection{Changes over time}\label{changes-over-time}}

Alkire, Roche, and Vaz (2017) proposed a four measures to assess pro-poor multidimensional
poverty reduction between two periods of time using repeated
cross-sectional data. The following measures are explained using the
Adjusted Headcount Ratio, however, all the changes-over-time measures
can be extend to the aforementioned partial measures.

Let \(t^1\) and \(t^2\) denote the initial and final period, respectively.
Then, \(M_0^{t^1}\) and \(M_0^{t^2}\) are their corresponding Adjusted
Headcounts Ratio. Note that for comparability purposes, these two
poverty measures must have the same set of parameters (indicators,
weights, deprivations and poverty cutoffs).

We define the \textit{absolute rate of change} (\(\Delta\)) as the
difference in \(M_0\) between the final \(t^2\) and the initial period
\(t^1\):

\$\$ \textbackslash Delta(M\_0)= M\_0\^{}\{t\^{}2\} - M\_0\^{}\{t\^{}1\}\$\$\$\$

The \textit{relative rate of change} (\(\delta\)) is defined as the
difference in \(M_0\) as a percentage of the initial poverty level \(t^1\),
i.e.,

\begin{equation}
\delta(M_0)= \frac{M_0^{t^2} - M_0^{t^1}}{M_0^{t^1}} \times 100
\end{equation}

On the other hand, the annualized versions of these two measures are
used in order to compare changes over time across countries with
different periods of reference. The
\textit{annualized absolute rate of change} (\(\bar{\Delta}\)) is the
absolute rate of change as defined above divided by the difference in
the two time periods:

\begin{equation}
\bar{\Delta}(M_0)= \frac{M_0^{t^2} - M_0^{t^1}}{(t^2 - t^1)}
\end{equation}

Finally, the \textit{annualized relative rate of change}
(\(\bar{\delta}\)) is defined as the compound rate of reduction in \(M_0\)
per year between the initial and the final periods, i.e.,
\begin{equation}\label{eq:ann_rel_change}
\bar{\delta}(M_0)= \left[ \left( \frac{M_0^{t^2}}{M_0^{t^1}} \right)^{\frac{1}{t^2 - t^1}} - 1 \right] \times 100
\end{equation}

An explanation of the usage of the main functions for estimating AF
measures and their changes over time using the \pkg{mpitbR} package is
provided in the following section.

\hypertarget{overview-of-package}{%
\section{\texorpdfstring{Overview of \pkg{mpitbR} package}{Overview of  package}}\label{overview-of-package}}

This toolbox package consists mainly of two functions,
\texttt{mpitb.set} and \texttt{mpitb.est}, which are the two key tools of the original Stata package. In addition, \pkg{mpitbR} includes functions from the build-in R package \pkg{stats} \texttt{coef}, \texttt{confint}, and \texttt{summary}. Last but not least, \pkg{mpitbR} mainly depends upon \pkg{survey} package as we will see below.

\hypertarget{mpitb.set-function}{%
\subsection{\texorpdfstring{\texttt{mpitb.set} function}{mpitb.set function}}\label{mpitb.set-function}}

In this first function, users specify the relevant information for
estimating the MPI. For instance, they provide the the data, the deprivation indicators for the MPI, and other optional auxiliary character arguments to label the project, such as its name and a brief description.

The usage and input arguments of function \texttt{mpitb.set} are summarized as
follows:

\begin{verbatim}
mpitb.set(data, indicators, ..., 
          name = "unnamed", desc = "desc.")
\end{verbatim}

The first argument should be a dataset containing information about the
household survey. To consider the survey design, users should specify
previously the survey structure using \texttt{svydesign} function from
\pkg{survey} package and pass a ``survey.design2''-class object to \texttt{data}
argument. If \texttt{data} is an object of the ``data.frame'' class, it is
coerced to a ``survey.design2''-class object assuming simple random
sampling design.

The \texttt{indicators} argument contains the name of the deprivation
indicators corresponding to the columns names in \texttt{data}. There are two
different ways to pass this argument: a list or a character vector. If
it is a list, the user will define the set of dimensions and their
corresponding indicators. Each element of the list represents a
dimension containing a character vector with the name of the indicators
associated to that dimension. This way is useful for automatically
calculating equal nested weights in the subsequent estimations. At most
10 dimensions are allowed. If \texttt{indicators} is a character vector, that
is, they are not grouped by dimensions. In this case, it is advisable to pass the
weights of each indicator manually with a numeric vector when estimating the AF measures. Otherwise, all indicators
will weight equally.

It is highly important to mention that it is assumed that the names of
the indicators in \texttt{data} are the columns of the deprivation matrix
\(\mathbf{g}^0\). In other words, the column names in \texttt{data} corresponding
to \texttt{indicators} should be binary variables. Another caveat that is worth
mentioning is that these columns should not contain any missing value.
The R \pkg{survey} package supports missing values for calculating the
point estimates but it would not be able to calculate the standard error
and, therefore, the confidence intervals.

Finally, \texttt{name} and \texttt{desc} arguments are useful for identifying each
setting while working in a multidimensional poverty measurement and
analysis project. The former is the project name and serves as an ID of
the setting. It is recommended to use short names (at most 10 characters
are permitted) while the latter is a character containing a brief
description.

The output of this function is a ``mpitb\_set''-class object with the
survey.design data, the name of the indicators and the setting name and
description as attributes.

\hypertarget{mpitb.est-function}{%
\subsection{\texorpdfstring{\texttt{mpitb.est} function}{mpitb.est function}}\label{mpitb.est-function}}

Once declared the relevant information for the multidimensional poverty
measurement and analysis project, in the \texttt{mpitb.est} function all the
parameters and the desired measures to be calculated are specified.

\begin{verbatim}
mpitb.est(set, klist = NULL, weights = "equal",
          measures = c("M0", "H", "A"), 
          indmeasures = c("hd", "hdk", "actb", "pctb"),
          indklist = NULL,  over = NULL,  ...,  
          cotyear = NULL,  tvar = NULL,  
          cotmeasures = c("M0", "H", "A", "hd", "hdk"), 
          ann = FALSE,  cotklist = NULL,  
          cotoptions = "total", 
          noraw = FALSE,  nooverall = FALSE, 
          level = 0.95,  
          multicore = getOption("mpitb.multicore"),
          verbose = TRUE)
\end{verbatim}

This function is a S3 method for ``mpitb\_set''-class objects declared in
\texttt{set} argument. Users pass the vector of poverty cut-offs (\(k\)) in \texttt{klist} argument in
percentage format, i.e., numbers between 1 and 100. Information about
the weighting scheme is defined in \texttt{weights} argument. By default it
calculates the nested equal weights for each indicator. It can be a
numeric vector of values between 0 and 1 such that all sum up to 1. Its
values should match the order of the indicators in \texttt{mpitb.set}.

Arguments \texttt{measures} and \texttt{indmeasures} control the AF measures to be
estimated. The former includes the MPI (or \texttt{"M0"}) and its disaggregation
by incidence (\texttt{"H"}) and intensity (\texttt{"A"}) of poverty. The latter
comprehend the rest of the indicator-specific AF measures such as the
uncensored and censored indicators headcount ratios (\texttt{"hd"}and
\texttt{"hdk"}) and their absolute and proportional contribution to overall
poverty (``\texttt{actb"} and \texttt{"pctb"}). It is possible to specify a poverty
cut-offs vector for these indicators-specific measures in \texttt{indklist}. If
this argument is \texttt{NULL}, it is equal to \texttt{klist}. By default, all these
measures are calculated, however, users can define the desired measures
to save time.

If any of these arguments is \texttt{NULL}, \texttt{mpitb.est()} skips these measures.
This avoids calculating unnecessary estimations. For instance,
if \texttt{measures\ =\ c("M0")} and \texttt{indmeasures\ =\ NULL}, only the MPI will be
estimated.

To specify the population subgroups (e.g., living area, sex, etc.) and
estimate the disaggregated measures by each level of the subgroup, the
users should pass the column names of the population subgroups in the
data using \texttt{over} argument. If \texttt{NULL}, the measures are estimated
using all the observations.

\hypertarget{changes-over-time-1}{%
\subsubsection{Changes over time}\label{changes-over-time-1}}

To analyze how multidimensional poverty changes over time, a variable to
index the time or survey round is needed. Argument \texttt{tvar} is a character
with the column name that references the time period. If annualized
measures are to be estimated, it is required the values of the years of
each round. \texttt{cotyear} argument is a character with the column name that
have information about the years (decimal digits allowed).

The \pkg{mpitbR} permits estimating the changes over time of each AF
measure for every population subgroup. This is declared in \texttt{cotmeasures}
argument. By default, it includes all AF measures but users can modify
it to save time. As with the indicator-specific measures, it is possible
to specify a poverty cut-offs vector for the changes over time in
\texttt{cotklist}. If \texttt{NULL}, it is equal to \texttt{klist}. Here it is worth
mentioning that the standard errors of changes-over-time measures are estimated using Delta method
(detailed in \texttt{svycontrast()} function from \pkg{survey} package).

If there are more than two survey rounds, \texttt{cotoptions} is used whether
to estimate the changes over the total period of observation or
year-to-year changes. To do the former, \texttt{cotoptions\ =\ "total"} whereas
for the latter case, \texttt{cotoptions\ =\ "insequence"}.

\hypertarget{other-arguments}{%
\subsubsection{Other arguments}\label{other-arguments}}

Some additional logical arguments are included to avoid unnecessary
estimations. \texttt{ann} is a control argument. If \texttt{TRUE}, annualized measures
are calculated. If \texttt{cotyear} is passed, \texttt{ann} is automatically set to
\texttt{TRUE}. If \texttt{cotyear} is not \texttt{NULL} and \texttt{ann} is \texttt{FALSE}, only
non-annualized measures are estimated. In the case that only annualized
measure are under study, the user can switch \texttt{noraw} to \texttt{TRUE} to avoid
estimating non-annualized changes.

On the other hand, the users specify the population subgroups of
interest in \texttt{over}. If \texttt{nooverall\ =\ TRUE}, estimations overall data
(e.g., national-level) are not calculated.

Finally, by default confidence intervals are estimated using a
confidence level of 95\%. Users can change this in \texttt{level} argument. The
confidence intervals are estimated considering measures as proportions
using \texttt{svyciprop()} function from \pkg{survey} package (it uses the
``logit'' method which consists of fitting a logistic regression model and
computes a Wald-type interval on the log-odds scale, which is then
transformed to the probability scale).

\pkg{mpitbR} includes the possibility to do parallel calculations over
all the measures and poverty cut-offs with the logical argument
\texttt{multicore}. The package uses Forking method for parallelization. Hence,
this option is only available on Unix-like systems.

\hypertarget{other-functions}{%
\subsection{Other functions}\label{other-functions}}

The output of \texttt{mpitb.est()} function mirrors the original Stata package.
It is a two-elements list where each one is a \texttt{data.frame} containing
all the estimations with the same format as the so-called Stata package.
Then, users can apply functions such as \texttt{coef}, \texttt{confint}, and \texttt{summary}.

Since the output data may contains multiple AF measures, by cut-offs and
levels of the subgroups and possibly by different time periods.
Therefore, for user-friendliness purposes, it is required to filter
data. For instance, all these methods are only available for one AF
measure.

\texttt{summary} function performs a t-test over the estimates inheriting the
confidence level in \texttt{level} argument of \texttt{mpitb.est()} function. The
\texttt{summary} function is particularly relevant for the changes-over-time
estimates where the user can infer if there have been or not a pro-poor
multidimensional poverty reduction between two periods of time. In the
case of changes over time, \texttt{summary} is available for one AF measure,
one change-over-time measure and one poverty cut-off.

The next section provides examples of how to implement these functions
in practice using a household survey.

\hypertarget{applications}{%
\section{Applications}\label{applications}}

This section includes illustrative examples for i) a traditional
exercise of cross-sectional estimations for a single country, ii) how to use options of the pacakge functions to save time and avoid unnecessary estimations, iii) using alternative
weighting schemes, and, if data permits, iv) study changes over time.

With the view of comparability, all the examples use the \texttt{syn\_cdta}
which is a synthetic data used in the Stata package examples
(Suppa 2023).

\hypertarget{estimate-af-measures-for-a-single-country-in-a-single-year}{%
\subsection{Estimate AF measures for a single country in a single year}\label{estimate-af-measures-for-a-single-country-in-a-single-year}}

First, we load the \pkg{survey} and \pkg{mpitbR} packages.

\begin{verbatim}
library(survey)
library(mpitbR)
\end{verbatim}

Table
\ref{tab:svydata-tab-static}
prints the first few rows of the data set variables.

\label{tab:svydata-tab-interactive}A synthetic household survey data

d\_nutr

d\_cm

d\_satt

d\_educ

d\_elct

d\_sani

d\_wtr

d\_hsg

d\_ckfl

d\_asst

area

region

stratum

psu

weight

year

t

0

0

1

0

0

0

0

1

0

0

1

4

401

401000

1

2010

1

0

0

0

0

1

0

0

1

1

1

1

1

104

104003

1

2010

1

0

0

0

0

1

0

0

0

0

0

1

20

2002

2002005

1

2010

1

0

0

0

0

0

0

0

1

0

0

1

20

2004

2004002

1

2010

1

0

0

1

0

1

0

0

1

0

0

1

18

1805

1805000

1

2010

1

0

0

1

0

1

1

0

1

0

1

0

18

1803

1803001

1

2010

1

\begin{table}

\caption{\label{tab:svydata-tab-static}A synthetic household survey data}
\centering
\fontsize{7}{9}\selectfont
\begin{tabular}[t]{r|r|r|r|r|r|r|r|r|r|l|l|r|r|r|r|r}
\hline
d\_nutr & d\_cm & d\_satt & d\_educ & d\_elct & d\_sani & d\_wtr & d\_hsg & d\_ckfl & d\_asst & area & region & stratum & psu & weight & year & t\\
\hline
0 & 0 & 1 & 0 & 0 & 0 & 0 & 1 & 0 & 0 & 1 & 4 & 401 & 401000 & 1 & 2010 & 1\\
\hline
0 & 0 & 0 & 0 & 1 & 0 & 0 & 1 & 1 & 1 & 1 & 1 & 104 & 104003 & 1 & 2010 & 1\\
\hline
0 & 0 & 0 & 0 & 1 & 0 & 0 & 0 & 0 & 0 & 1 & 20 & 2002 & 2002005 & 1 & 2010 & 1\\
\hline
0 & 0 & 0 & 0 & 0 & 0 & 0 & 1 & 0 & 0 & 1 & 20 & 2004 & 2004002 & 1 & 2010 & 1\\
\hline
0 & 0 & 1 & 0 & 1 & 0 & 0 & 1 & 0 & 0 & 1 & 18 & 1805 & 1805000 & 1 & 2010 & 1\\
\hline
0 & 0 & 1 & 0 & 1 & 1 & 0 & 1 & 0 & 1 & 0 & 18 & 1803 & 1803001 & 1 & 2010 & 1\\
\hline
\end{tabular}
\end{table}

The first ten columns in our data set contains the ten global MPI
indicators as binary variables. Following the order of the columns,
these names stand for: Nutrition, Child Mortality, School Attendance,
Education, Electricity, Sanitation, Water, Housing, Cooking Fuel, and
Assets (see UNDP and OPHI 2022, for more details). In addition, there are
two columns of population subgroups to decompose the MPI:\texttt{area} and
\texttt{region}.

The data set consists of two surveys. The variables \texttt{year} and \texttt{t}
provide information about the year and the number of each survey round.
The survey design is defined by the variables that contains the primary
sampling unit, the weights and the strata of each observation (\texttt{psu},
\texttt{weights} and \texttt{stratum}, respectively)

Prior to any calculations, it is necessary to define the design of our
survey data. This is highly important because poverty indices are
estimated using household surveys and accounting for the complex
survey design ensures valid statistical inferences, accurate variance
estimation, and representative estimates.

We restrict the analysis for a single year. In this case, the first
round of the survey (i.e., \texttt{t\ ==\ 1}). All the information of the survey
structure of our data is declared using \texttt{svydesign} function from
\pkg{survey} package. Our data contains some missing values in our
indicators. Hence, we pass our data to the \texttt{svydesign} function omitting
them.

\begin{verbatim}
# Subset data
syn_cdta1 <- subset(syn_cdta, t == 1)
# Drop NA in indicators columns
syn_cdta1 <- na.omit(syn_cdta1)
# Define survey design
svydata1 <- svydesign(id=~psu, weights = ~weight, strata = ~stratum, data = syn_cdta1)
\end{verbatim}

To specify the MPI estimation settings, we use the \texttt{mpitb.set} function.
Apart from the survey data, another required argument is \texttt{indicators},
in which we select the indicators columns with the possibility to organize them in a set of
dimensions. This can be defined first in a list and the passed to the function as
follows.

\begin{verbatim}
indicators <- list(hl = c("d_nutr","d_cm"),
                   ed = c("d_satt","d_educ"),
                   ls = c("d_elct","d_sani","d_wtr","d_hsg","d_ckfl","d_asst"))

set.trial01 <- mpitb.set(svydata1, indicators = indicators, 
                         name = "trial01", desc = "pref. spec")
\end{verbatim}

Note that the indicators are grouped in three dimensions (\texttt{hl}, \texttt{ed}, and
\texttt{ls}) that stand for Health, Education and Living Standards (the three
dimensions included in the global MPI estimations). Finally, we defined a
short name (\texttt{"trial01"}) and description (\texttt{"pref.\ spec"}) to our
project.

All the information about the specification of the multidimensional
poverty measurement is stored in the \texttt{set.trial01} object, which is of the class
``mpitb\_set'' and then passed to the \texttt{mpitb.est} S3 method in order to
compute the indices.

For this first round assume we want to estimate all the AF measures.
These are all the aggregate measures (\(M_0\), \(H\), and \(A\)) and the
indicator-specific measures (\(h_d\), \(h_d(k)\), \(actb\), and \(pctb\)). Also,
we prefer a equal-nested weighting scheme (equal weights for all
dimensions and equal indicator weights within dimensions) and for each
measure we set three poverty cut-offs: \(20\%\), \(33\%\), and \(50\%\).
Finally, we also want to calculated the disaggregated measure by the
different subnational regions and living areas (urban and rural),
accounting for the complex design of the survey.

All this information is specified in the \texttt{mpitb.est} function as
follows:

\begin{verbatim}
est.trial01 <- mpitb.est(set = set.trial01, klist = c(20, 33, 50),
                         weights = "equal", 
                 measures = c("M0","H","A"), 
                 indmeasures = c("hd", "hdk", "actb", "pctb"),
                 over = c("area","region"))
\end{verbatim}

The ``mpitb\_set''-class object is passed in \texttt{set} argument. The poverty
cut-offs in \texttt{klist} and the population subgroups in \texttt{over}. The
equal-nested weighting scheme is specified in \texttt{weights} argument passing
the character \texttt{"equal"}. The aggregate measures and the
indicator-specific measure in \texttt{measures} and \texttt{indmeasures} respectively.
However, by default all these measures and equal-nested weights are
calculated. Hence, this arguments can easily be omitted in this case.

Users can verified the specification of their project with the
\texttt{mpitb.est} messages. It reports 1) the function call to verified correct
arguments assignment, 2) the dimensions with their assigned
indicators and their corresponding weights, 3) which measures are being
estimated, 4) the parameters of the estimation (number of poverty cut-offs and
subgroups), and 5) other features such as the confidence level and if
parallel estimation have been used. All these messages can be suppressed
using \texttt{verbose\ =\ FALSE} in \texttt{mpitb.est} function.

The \texttt{mpitb.est} function returns a two-element list (``mpitb\_est''-class),
where each element is a data frame containing all the estimates. The
first (`lframe') include all the cross-sectional estimates for each
level of analysis, whereas the second (`cotframe') all the changes over
time measures for each level of analysis. In this first example,
`cotframe' is \texttt{NULL}.

For this instance, the \texttt{lframe} first rows are as depicted:

\begin{verbatim}
head(est.trial01$lframe)
\end{verbatim}

Each row represent an estimate from each measure and population subgroup. In this example, we have 2507 estimates.
These data frames resemble the output of the so-called Stata package.
The first columns include most important features such as the point
estimate (\texttt{"b"}) with the corresponding standard error (\texttt{"se"}) that
account for the complex survey design and the lower and upper confidence
limits (\texttt{"ll"} and \texttt{"ul"}) calculated with a given confidence level.
Each row represent an estimate of a \texttt{"measure"} for a specific
\texttt{"indicator"}, if applied, given a cut-off \texttt{"k"} for each subgroup level
of analysis, which can be tracked in the columns of \texttt{"loa"} and
\texttt{"subg"}.

Since the elements of the list are data frames, users can easily subset
the `lframe' or `cotframe' elements to examine point estimates or
confidence intervals of certain subgroups and measures of interest. For
instance, assume we want to see the incidence \(H\) confidence intervals
for the \texttt{"area"} subgroups for the cut-off \(33\%\), the line code goes as
follows:

\begin{verbatim}
confint(subset(est.trial01$lframe, measure == "H" & loa == "area" & k == 20))
\end{verbatim}

Finally, \texttt{summary} calculates a t-test statistic for each point
estimate, with the corresponding p-value and significance level to infer
if it is statistically different to zero. Such analysis is more
practical when it comes to make inference of any change-over-time measure. An
example of this will be provided below.

\hypertarget{mpitb.est-arguments-to-avoid-unnecessary-estimations}{%
\subsection{\texorpdfstring{\texttt{mpitb.est} arguments to avoid unnecessary estimations}{mpitb.est arguments to avoid unnecessary estimations}}\label{mpitb.est-arguments-to-avoid-unnecessary-estimations}}

As it is possible to observe from the previous example, the amount of
estimates can soar for a small number of parameters (poverty cut-offs,
population subgroups, etc.). Hence, in order to avoid unnecessary
estimations and save time, it is important to determine which are the
measures to be prioritized.

For example, although the deprivation scores, \(c_i\) and \(c_i(k)\) , are
real-valued functions, for fixed values of weights and \(k\) , they will
assume a finite number of values. Hence, it is advisable to be aware of
which values of \(k\) are included to avoid estimating measures that will
yield the same results. Furthermore, this is specially important if we
incorporate different population subgroups into the analysis since adding subgroups in the main source of the increasing number of estimations.

In this package, it is possible to control which AF measures we want to
estimate and to specify a separate list of poverty cut-off values for
the aggregate measures and the indicator-specific measures with \texttt{klist}
and \texttt{indklist} arguments.

In the following lines of code, we run the same last example with
few changes. We rule out from the analysis the \(H\) and \(A\)
measures and the contribution measures of each indicator. On the other
hand, we specify a different poverty cut-off for the indicator specific
measures and we avoid estimating national level estimates
setting \texttt{nooverall} argument to \texttt{TRUE}.

\begin{verbatim}
mpitb.est(set = set.trial01, klist = c(20, 33, 50),
          weights = "equal", measures = c("M0"), 
          indmeasures = c("hd", "hdk"), indklist = c(33), 
          over = c("area","region"), nooverall = TRUE)
\end{verbatim}

The number of estimates have been reduced drastically in
comparison to the former example. In the previous estimations there were a total of 2507 estimates whereas in this case 946 estimates.

\hypertarget{specify-alternative-weighting-schemes}{%
\subsection{Specify alternative weighting schemes}\label{specify-alternative-weighting-schemes}}

In the previous examples we have assumed equal-nested weights across
dimensions and indicators. In other words, equal weights for all
dimensions and equal indicator weights within dimensions. We have
specified this by passing the indicators grouped in a list to \texttt{mpitb.set} function and setting \texttt{weights\ =\ "equal"}, which
is the default value in \texttt{mpitb.est} function.

There is an alternative way to do this, which is passing a numeric vector to
\texttt{weights} argument with the values of the weights corresponding to each
indicator exemplified here below.

\begin{verbatim}
mpitb.est(set = set.trial01, klist = c(33), 
          measures = c("M0"), indmeasures = NULL,
          weights = c(1/6, 1/6, 1/6, 1/6, 
                      1/18, 1/18, 1/18, 1/18, 1/18, 1/18))
\end{verbatim}

Now assume we assign a weight of \(50\%\) to Health dimension and \(25\%\)
to the rest of dimension. If we take equal weights in each indicator
within each dimension, the vector of weights will be specify as follows:

\begin{verbatim}
mpitb.est(set = set.trial01, klist = c(33), 
          measures = c("M0"), indmeasures = NULL,
          weights = c(1/4, 1/4, 1/8, 1/8, 
                      1/24, 1/24, 1/24, 1/24, 1/24, 1/24))
\end{verbatim}

Another convenient feature of this workflow is that users can easily
evaluate the effects of adding or dropping some indicators, merging two
indicators, and comparing different deprivation thresholds and
alternative selections of indicators. Such analysis, jointly with
specifying alternative weighting schemes, is often used during the
construction process of an MPI. The following lines of code show the
same first example framework but dropping electricity indicator from the
analysis.

\begin{verbatim}
indicators <- list(hl = c("d_nutr","d_cm"),
                   ed = c("d_satt","d_educ"),
                   ls = c("d_sani","d_wtr","d_hsg","d_ckfl","d_asst"))

set.trial02 <- mpitb.set(syn_cdta, indicators = indicators,
                 name = "trial02", desc = "w/o electricity")

est.trial02 <- mpitb.est(set = set.trial02, klist = c(20), weights = "equal", 
                 measures = c("M0","H","A"), 
                 indmeasures = c("hd", "hdk", "actb", "pctb"),
                 over = c("area","region"))
\end{verbatim}

\hypertarget{estimate-af-measures-for-a-single-country-in-several-years}{%
\subsection{Estimate AF measures for a single country in several years}\label{estimate-af-measures-for-a-single-country-in-several-years}}

In order to analyse changes over time, we need a sequence of survey
rounds for the same population. This workflow assumes that the microdata
from the different survey are appended with a column that serves as an
identifier. For instance, in \texttt{syn\_cdta} data the variable \texttt{t} index the
two available survey rounds (1 and 2).

In our first example, we subset the data selecting the rows
corresponding to the first year round \texttt{t\ ==\ 1}. Now we will use the
entire data set to specify our survey design with the difference that we
have to pass which is the column that indexes the survey rounds in the
\texttt{tvar} argument of \texttt{mpitb.est} function.

\begin{verbatim}
# Drop NA in indicators columns
syn_cdta <- na.omit(syn_cdta)
# Define survey design
svydata <- svydesign(id=~psu, weights = ~weight, strata = ~stratum, data = syn_cdta)
\end{verbatim}

\begin{verbatim}
indicators <- list(hl = c("d_nutr","d_cm"),
                   ed = c("d_satt","d_educ"),
                   ls = c("d_elct","d_sani","d_wtr","d_hsg","d_ckfl","d_asst"))

set.trialCOT <- mpitb.set(svydata, indicators = indicators, name = "trialCOT", desc = "Estimate changes over time")
\end{verbatim}

\begin{verbatim}
est.trialCOT <- mpitb.est(set = set.trialCOT, klist = c(1, 33, 50),
                 indmeasures = NULL,
                 over = c("region"), tvar = "t", cotyear = "year")
\end{verbatim}

\hypertarget{summary}{%
\section{Summary}\label{summary}}

We have presented the main functions for estimating multidimensional
poverty indices (MPI) with \pkg{mpitbR} package which aims to provide a
thorough but user-friendly framework for both academics and
practitioners of multidimensional poverty measurement.

Because the toolbox has been developed in the context of the global MPI,
it is also tailored to its needs, whether in terms of the underlying
data, the quantities produced out of the box, or the related forms of
analysis.

\begin{center}\rule{0.5\linewidth}{0.5pt}\end{center}

Multidimensional poverty measurement and analysis is, however, an active
field of research, where new measures, analyses, and other
methodological innovations are still proposed and discussed. mpitb may
already be useful for such endeavors and take some load off of
researchers working these topics.

The very nature of mpitb as a toolbox seeks to allow for further
features, novel analyses, and additional tools being added in the
future. Likewise, adding support for novel complementary measures within
the AF framework, for example, for the analysis of inequality among the
poor (Alkire and Foster 2019) seems natural. Extensions along these
lines may be implemented directly into mpitb est. Other types of
analyses, however, may require one or more tools on their own, such as a
panel-data-based analysis within the AF framework (for example, Alkire
et al.~{[}2017a{]}, Suppa {[}2018{]}). Standalone tools may also be needed for
the analysis of pairwise robust comparisons, which examines country
orderings in terms of their poverty indices (Alkire and Santos 2014;
Alkire et al.~2022a), or the recently proposed modeling framework for
computing projections of multidimensional poverty (Alkire et al.
Forthcoming). Aside from the implementation of genuine methodological
innovations, one may also consider convenience tools, which, for
instance, help to compare different measures using specific tabulations
or visualizations during the trial stage. Future developments, however,
depend on many factors, including user needs, further progress in
research, and available resources.

\hypertarget{acknowledgements}{%
\section{Acknowledgements}\label{acknowledgements}}

I would like to express my sincere thanks both to Rodrigo García
Arancibia and José Vargas, my mentors, for their in valuable guidance,
feedback, and encouragement throughout the development of this package.
I also wish to acknowledge Nicolai Suppa for his generous support and
enthusiasm in adapting his Stata package to R users. Without his
constructive suggestions this package would not have been possible.
Finally, I would like to thank every instructor from the OPHI Summer
School 2022, for their thorough lectures on the Alkire-Foster method for
multidimensional poverty measurement and the provision of the Stata
source code which mainly motivated this project.

\hypertarget{references}{%
\section*{References}\label{references}}
\addcontentsline{toc}{section}{References}

\hypertarget{refs}{}
\begin{CSLReferences}{1}{0}
\leavevmode\vadjust pre{\hypertarget{ref-MPIpolicytool}{}}%
Alkire, Sabina. 2021. {``Multidimensional Poverty Measures as Policy Tools.''} In, edited by \& R. Lepenies V. Beck H. Hahn, 197--214. Philosophy and Poverty 2. Springer.

\leavevmode\vadjust pre{\hypertarget{ref-af11}{}}%
Alkire, Sabina, and J. E. Foster. 2011. {``Counting and Multidimensional Poverty Measurement.''} \emph{Journal of Public Economics} 95 (7): 476--87. \url{https://doi.org/10.1016/j.jpubeco.2010.11.006}.

\leavevmode\vadjust pre{\hypertarget{ref-alkire2015multidimensional}{}}%
Alkire, Sabina, James Foster, Suman Seth, Maria Emma Santos, José Manuel Roche, and Paola Ballon. 2015. \emph{Multidimensional Poverty Measurement and Analysis}. Oxford University Press. \url{https://doi.org/10.1093/acprof:oso/9780199689491.001.0001}.

\leavevmode\vadjust pre{\hypertarget{ref-COT_mpi}{}}%
Alkire, Sabina, José Manuel Roche, and Ana Vaz. 2017. {``Changes over Time in Multidimensional Poverty: Methodology and Results for 34 Countries.''} \emph{World Development} 94: 232--49. https://doi.org/\url{https://doi.org/10.1016/j.worlddev.2017.01.011}.

\leavevmode\vadjust pre{\hypertarget{ref-survey_paper}{}}%
Lumley, Thomas. 2004. {``Analysis of Complex Survey Samples.''} \emph{Journal of Statistical Software} 9 (8): 1--19. \url{https://doi.org/10.18637/jss.v009.i08}.

\leavevmode\vadjust pre{\hypertarget{ref-mpitb_Stata}{}}%
Suppa, Nicolai. 2023. {``{m}pitb: A Toolbox for Multidimensional Poverty Indices.''} \emph{The Stata Journal} 23 (3): 625--57. \url{https://doi.org/10.1177/1536867X231195286}.

\leavevmode\vadjust pre{\hypertarget{ref-gMPI-report2022}{}}%
UNDP, and OPHI. 2022. {``\href{}{2022 Global Multidimensional Poverty Index (MPI)}.''} \emph{UNDP (United Nations Development Programme)}.

\end{CSLReferences}

\bibliography{RJreferences.bib}

\address{%
Ignacio Girela\\
CONICET - Universidad Nacional de Córdoba\\%
Facultad de Ciencias Económicas\\ Córdoba, Argentina\\
%
\url{https://www.eco.unc.edu.ar/}\\%
\textit{ORCiD: \href{https://orcid.org/0000-0003-3297-3854}{0000-0003-3297-3854}}\\%
\href{mailto:ignacio.girela@unc.edu.ar}{\nolinkurl{ignacio.girela@unc.edu.ar}}%
}
